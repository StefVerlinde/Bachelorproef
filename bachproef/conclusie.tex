%%=============================================================================
%% Conclusie
%%=============================================================================

\chapter{Conclusie}
\label{ch:conclusie}

% TODO: Trek een duidelijke conclusie, in de vorm van een antwoord op de
% onderzoeksvra(a)g(en). Wat was jouw bijdrage aan het onderzoeksdomein en
% hoe biedt dit meerwaarde aan het vakgebied/doelgroep? 
% Reflecteer kritisch over het resultaat. In Engelse teksten wordt deze sectie
% ``Discussion'' genoemd. Had je deze uitkomst verwacht? Zijn er zaken die nog
% niet duidelijk zijn?
% Heeft het onderzoek geleid tot nieuwe vragen die uitnodigen tot verder 
%onderzoek?

In dit onderzoek werd zowel een antwoord gegeven op de onderzoeksvraag alsook op verschillende deelvragen. De onderzoeksvraag was: \emph{Hoe kan Ventigrate op een professionele manier APIs en clients opzetten beveiligd door Azure active directory aan de hand van rolling keys?}. \newline \newline 
Uit dit onderzoek werd geleerd hoe JWT bearer authenticatie in zijn werk gaat. Hieruit werd geconcludeerd dat een JWT token of JSON web token bestaat uit drie delen, een header, payload en signature. Wat opviel is dat de header vier claims bevat die essentieel zijn voor het valideren van de token. De payload bevat meer claims maar niet allemaal verplicht. Tot deze claims behoren bijvoorbeeld de audiance of de roles maar die kan ook de openId claim bevatten wanneer er ook aan authenticatie gedaan wordt. Het signature deel kan worden gebruikt om de authenticiteit van het token te valideren, zodat het door de applicatie kan worden vertrouwd. \newline \newline
Op de deelvragen hoe je een API moet beveiligen met client credential of implicit grant wordt antwoord gegeven op een schematische wijze. Aan de hand van enkele stappen wordt een schema opgebouwd om zo de onderliggende werking te verduidelijken. Uit beide onderzoeken zijn gelijkaardige resultaten gevonden. Op het gebruik van de access token na verloopt de flow tussen Azure active directory, api en client gelijkaardig. In beide onderzoeken wordt vooral aandacht besteed aan de verschillende termen en gegevens die terug te vinden zijn bij het doorlopen van de stappen, zoals bijvoorbeeld client ID, resource ID en client secret. Het gebruik van deze waarden moet discreet en juist gebeuren. Er wordt in beide gevallen ook verwezen naar een gemaakte proof of concept als voorbeeld. Deze is echter enkel bedoeld voor demonstratiedoeleinden en is zeker niet aan te raden voor development. De reden hiervoor is dat de discrete waarden zoals, client secret, opgeslagen worden in een json-file in de applicatie. Wanneer we deze proof of concept in development zouden gebruiken, zouden we deze opslaan in een key vault of user secret.\newline\newline
Wanneer wordt gekeken naar hoe een API kan aangesproken worden met authorization code grant valt op dat de schematische voorstelling gelijkaardig is aan de voorgaande voorbeelden. Het verschil dat hier opvalt is de extra call naar AAD voor een athorize code.\newline\newline
Bij Azure key rollover is het belangrijk om te weten welk type situatie je hebt. Bij dit onderzoek zien we dat er verschillende situaties zijn waar clients anders reageren op een key rollover van Azure. In de voorbeelden die bij de vorige deelvragen gebruikt werden hadden we te maken met een clientapplicatie die toegang heeft tot een resource en een webapplicatie die gebruik maakte van JWT Bearer authentication. In de eerste situatie moet er geen rekening gehouden worden met key rollovers aangezien ze geen resources beschermen en hierdoor inspecteren ze de token niet. Bij JWT Bearer token authentication wordt de logica om een key rollover te verwerken al toegevoegd.\newline\newline
Kortom, de onderzoeksvraag kan worden beantwoord aan de hand van de verschillende antwoorden op de deelvragen. Ventigrate kan op een professionele manier APIs en clients opzetten en beveiligen met Azure active directory aan de hand van rolling keys door aan de hand van wat gewenst is het antwoord van de juiste situatie te implementeren. Terwijl dit gebeurt zal de lezer het onderwerp ook beter begrijpen en weten wat er zich achter de schermen afspeelt.

