%%=============================================================================
%% Samenvatting
%%=============================================================================

% TODO: De "abstract" of samenvatting is een kernachtige (~ 1 blz. voor een
% thesis) synthese van het document.
%
% Deze aspecten moeten zeker aan bod komen:
% - Context: waarom is dit werk belangrijk?
% - Nood: waarom moest dit onderzocht worden?
% - Taak: wat heb je precies gedaan?
% - Object: wat staat in dit document geschreven?
% - Resultaat: wat was het resultaat?
% - Conclusie: wat is/zijn de belangrijkste conclusie(s)?
% - Perspectief: blijven er nog vragen open die in de toekomst nog kunnen
%    onderzocht worden? Wat is een mogelijk vervolg voor jouw onderzoek?
%
% LET OP! Een samenvatting is GEEN voorwoord!

%%---------- Nederlandse samenvatting -----------------------------------------
%
% TODO: Als je je bachelorproef in het Engels schrijft, moet je eerst een
% Nederlandse samenvatting invoegen. Haal daarvoor onderstaande code uit
% commentaar.
% Wie zijn bachelorproef in het Nederlands schrijft, kan dit negeren, de inhoud
% wordt niet in het document ingevoegd.

\IfLanguageName{english}{%
\selectlanguage{dutch}
\chapter*{Samenvatting}
\lipsum[1-4]
\selectlanguage{english}
}{}

%%---------- Samenvatting -----------------------------------------------------
% De samenvatting in de hoofdtaal van het document

\chapter*{\IfLanguageName{dutch}{Samenvatting}{Abstract}}


In dit onderzoek worden alle aspecten omtrent Azure active directory bearer authentication besproken. De onderzoeksvraag luidt: "Hoe kan Ventigrate op een professionele manier APIs en clients opzetten beveiligd door Azure active directory aan de hand van rolling keys?". Het onderzoek gaat verder in op de OAuth authorisation en openID authentication frameworks en hun geschiedenis. Ook worden grant types en JWT bearer tokens besproken en tot slot wordt er ook een proof of concept bekeken voor het opbouwen van een secure API met een client die aan de hand van de client credential flow veilig de API kan aanspreken. Er wordt ook dieper ingegaan op een tweede proof of concept die aan de hand van de password grant een web applicatie gaat beveiligen.\newline\newline
Dit onderzoek is essentieel om meer te weten te komen over de werking van Azure active directory en bearer authentication. Dit onderwerp werd onderzocht omdat er in de bedrijfswereld veel vraag naar is. Er wordt steeds meer met cloud based authentication gewerkt aan de hand van identity servers.	\newline\newline
Om dit onderzoek tot een goed einde te brengen werd er voldoende informatie verzameld over de werking van de het OAuth framework. Zo werd de uitwisseling van de access token geschematiseerd om de flow tussen client, API en identity server te doorgronden. Ook werd er naar de geschiedenis gekeken om te weten waar het framework vandaan kwam en waarom dit de dag van vandaag zo populair is. Vooraleer de proof of concepts gemaakt werden, werd er informatie verzameld over Azure, meer specifiek over Azure active directory. Aan de hand van deze informatie konden twee proof of concept uitgebouwd worden om het onderzoek te staven. In de conclusie kan gelezen worden wat het verschil is tussen de verschillende grant types, hoe de access token anders gebruikt wordt aan de hand van welke grant type en wat er gebeurt bij een Azure key roll over. Een mogelijk vervolg op dit onderzoek zou een onderzoek naar identity server kunnen zijn, hoe en wat er nodig is om deze op te zetten en wat er achter de schermen gebeurt.