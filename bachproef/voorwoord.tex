%%=============================================================================
%% Voorwoord
%%=============================================================================

\chapter*{\IfLanguageName{dutch}{Woord vooraf}{Preface}}
\label{ch:voorwoord}

%% TODO:
%% Het voorwoord is het enige deel van de bachelorproef waar je vanuit je
%% eigen standpunt (``ik-vorm'') mag schrijven. Je kan hier bv. motiveren
%% waarom jij het onderwerp wil bespreken.
%% Vergeet ook niet te bedanken wie je geholpen/gesteund/... heeft
Het onderwerp OAuth aan de hand van rolling keys sprak mij erg aan aangezien ik deze termen al enkele malen was tegengekomen in onderzoek naar technieken en frameworks. Nadat ik deze termen eveneens was tegengekomen bij het lezen van een opdracht die ik voor een bedrijf moest uitvoeren, was ik ervan overtuigd dat dit onderwerp niet enkel zeer interessant zou zijn, maar ook vaak gebruikt zou worden in de dagelijkse developmentwereld. \newline\newline
Nadat ik kort het onderwerp had bekeken en hier enkele artikels over had gelezen, bleek dit nog interessanter te zijn dan ik had verwacht. Het framework OAuth is namelijk zo populair dat zo goed als iedereen hier al eens gebruik van gemaakt heeft zonder dit te beseffen. Heb je al eens ingelogd met Facebook of Google op een website van derden? Dan heb je de OAuth flow al aan den lijve ondervonden. \newline\newline
Ook interesseerde de combinatie met azure active directory rolling keys mij. Zeker na een tijdje mee te draaien in de bedrijfswereld, merk je dat azure extreem populair is binnen bedrijven en organisaties en dat dit platform zeer nuttige functionaliteiten heeft. \newline\newline
De combinatie van het populaire framework OAuth en het gebruik van een veel gebruikt platform zoals azure, daar wilde ik mij in verdiepen. Daarbij leek het mij ook voor bedrijven zeker de moeite om hier dieper onderzoek naar te doen. Ik besloot om voor mijn bachelorproef hier verder mee aan de slag te gaan. \newline\newline
Ik zou graag de lectoren Thomas Pollet en Jens Buysse willen bedanken voor het begeleiden van dit onderzoek. Ook bedank ik mijn ouders, zus, vriendin en vrienden voor de mentale steun die zij geboden hebben tijdens het schrijven van deze bachelorproef. Verder zou ik graag mijn stagebegeleider Benjamin De Clercq en collega's Arne Deruwe, Tim Van Roosbroeck, Karel Heydrickx, Nikola Invernizzi en Arne Vanhee willen bedanken voor de nodige uitleg en extra informatie.

