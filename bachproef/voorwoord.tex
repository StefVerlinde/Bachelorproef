%%=============================================================================
%% Voorwoord
%%=============================================================================

\chapter*{\IfLanguageName{dutch}{Woord vooraf}{Preface}}
\label{ch:voorwoord}

%% TODO:
%% Het voorwoord is het enige deel van de bachelorproef waar je vanuit je
%% eigen standpunt (``ik-vorm'') mag schrijven. Je kan hier bv. motiveren
%% waarom jij het onderwerp wil bespreken.
%% Vergeet ook niet te bedanken wie je geholpen/gesteund/... heeft
Het onderwerp OAuth aan de hand van rolling keys sprak mij erg aan aangezien ik deze termen al enkele malen was tegengekomen in onderzoek naar technieken en frameworks. Na dat ik deze termen eveneens was tegengekomen bij het lezen van een opdracht die ik voor een bedrijf moest uitvoeren was ik ervan overtuigd dat dit onderwerp niet enkel zeer interessant ging zijn, maar ook vaak gebruikt wordt in de dagelijkse development wereld. \newline\newline
Na kort het onderwerp te bekijken en hier enkele artikels over te lezen bleek dit nog interessanter te zijn dan ik had verwacht. Het framework OAuth is namelijk zo populair dat zo goed als iedereen hier al eens gebruik van gemaakt heeft zonder dit te beseffen. Heb je al eens ingelogd met facebook of google op een website van derden? Dan heb je de OAuth flow al aan de lijven ondervonden. \newline\newline
Ook interesseerde de combinatie met azure active directory rolling keys mij. Zeker na een tijdje mee te draaien in de bedrijfswereld merk je dat azure extreem populair is binnen bedrijven en organisaties en dat dit platform zeer nuttig functionaliteiten heeft. \newline\newline
De combinatie van de het populaire framwork OAuth en het gebruik van een veel gebruikt platform zoals azure leek mij erg interessant om mijzelf in te verdiepen. Daarbij leek het mij voor bedrijven zeker interessant om hier dieper onderzoek naar te doen. \newline\newline
Door deze redenen heb ik ervoor gekozen om voor mijn bachelorproef hier verder onderzoek naar te doen. \newline\newline
Ik zou graag de lectoren Thomas Pollet en Jens Buysse willen bedanken voor het begeleiden van dit onderzoek. Alsook mijn ouders, zus, vriendin en vrienden voor de mentale steun die zij geboden hebben in het verloop van deze bachelorproef. Verder zou ik graag mijn stagebegeleider Benjamin De Clercq en collegas Arne Deruwe, Tim Van Roosbroeck, Karel Heydrickx, Nikola Invernizzi en Arne Vanhee willen bedanken voor de nodige uitleg en extra informatie.

