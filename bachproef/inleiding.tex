%%=============================================================================
%% Inleiding
%%=============================================================================

\chapter{\IfLanguageName{dutch}{Inleiding}{Introduction}}
\label{ch:inleiding}
\section{\IfLanguageName{dutch}{Probleemstelling}{Problem Statement}}
\label{sec:probleemstelling}

De probleemstelling kan beschreven worden als een gebrekkige, onduidelijke of foutieve documentatie omtrent het onderwerp OAuth aan de hand van rolling keys op het internet. Het bedrijf Ventigrate verwacht dat ik dit probleem oplos door een volledige documentatie op te stellen rond OAuth aan de hand van rolling keys met bijhorende proof of concept. \newline\newline
Mijn doelgroep bestaat uit Ventigrate met opdrachtgevers Kevin Pelkmans en Yannick Borghmans, maar kan ook interessant zijn voor bedrijven en zelfstandigen die interesse hebben om hun autorisatie- en authenticatiesysteem professioneel op te bouwen of bij te stellen.

\section{\IfLanguageName{dutch}{Onderzoeksvraag}{Research question}}
\label{sec:onderzoeksvraag}

De onderzoeksvraag voor dit onderzoek kan duidelijk geformuleerd worden als: "Hoe kan Ventigrate op een professionele manier een OAuth-autorisatieserver opzetten aan de hand van azure active directory rolling keys?" 

\section{\IfLanguageName{dutch}{Onderzoeksdoelstelling}{Research objective}}
\label{sec:onderzoeksdoelstelling}

De doelstelling voor dit onderzoek is het opbouwen van een complete proof of concept met bijhorende documentatie voor het opzetten van een autorisatieserver. Dit onderzoek wordt als een succes beschouwd als Ventigrate met gebruik van deze documentatie een volledig en correct idee heeft over wat OAuth inhoudt en hoe dit samenspeelt met azure. Ook moet Ventigrate in staat zijn om samen met de proof of concept een autorisatieserver op te zetten voor het bedrijf zonder extra bronnen van derden te raadplegen.

\section{\IfLanguageName{dutch}{Opzet van deze bachelorproef}{Structure of this bachelor thesis}}
\label{sec:opzet-bachelorproef}

% Het is gebruikelijk aan het einde van de inleiding een overzicht te
% geven van de opbouw van de rest van de tekst. Deze sectie bevat al een aanzet
% die je kan aanvullen/aanpassen in functie van je eigen tekst.

De rest van deze bachelorproef is als volgt opgebouwd:

In Hoofdstuk~\ref{ch:stand-van-zaken} wordt een overzicht gegeven van de stand van zaken binnen het onderzoeksdomein, op basis van een literatuurstudie.

In Hoofdstuk~\ref{ch:methodologie} wordt de methodologie toegelicht en worden de gebruikte onderzoekstechnieken besproken om een antwoord te kunnen formuleren op de onderzoeksvragen.

% TODO: Vul hier aan voor je eigen hoofstukken, één of twee zinnen per hoofdstuk

In Hoofdstuk~\ref{ch:conclusie}, tenslotte, wordt de conclusie gegeven en een antwoord geformuleerd op de onderzoeksvragen. Daarbij wordt ook een aanzet gegeven voor toekomstig onderzoek binnen dit domein.